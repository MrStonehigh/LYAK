\chapter{Projektformulering}


Formålet med projektet er, at ramme et afstandsmæssigt "sweet-spot" for et basrefleks-kabinet, hvor porten behjælper yderligere konstruktiv interferens med overfladen den er i nærheden af, og dermed giver mulighed for at forstærke basgengivelsen yderligere ved de laveste frekvenser. 

Hertil ønskes en konklusion på, hvorvidt disse parametre vil gøre en psyko-akustisk forskel for brugeren, og hvorvidt basrefleks-portens position i kabinettet kan have en positiv indvirkning på den subjektive oplevelse af lydgenvgivelsen, såvel som det fysiske tryk for bas-gengivelsen. 



