\chapter*{Resume}

For at undgå akustisk kortslutning af en højtalerenhed, har man gennem historien fundet forskellige metoder til at modvirke dette fysiske fænomen. 
Den akustiske kortslutning forekommer ved de laveste frekvenser, hvor bølgelængden er meget stor ift. højtalerenhedens overfladeareal, og dermed er enheden ikke i stand til at danne differens i luftrykket mellem enhedens for- og bagside.

Ved at sætte enheden i et kabinet, kan man undgå denne akustiske kortslutning i høj grad, og ydermere udnytte kabinettets egen-resonans til at forbedre højtalerenhedens gengivelse af de lavere frekvenser. 

Denne resonans er beskrevet som det princip man kalder for ventileret kabinet, eller "basreflex-kabinet"; Et princip der benytter et rør i kabinettet som, ved korrekt afstemning, kan give en forstærkning ved resonans og dermed opnå et højere gain ved de laveste frekvenser. En billigere løsning end slave-kabinettet og mere effektiv løsning end det lukkede kabinet. 

Dette projekt har, foruden at designe et sådan kabinet, gået på at eksperimentere med at opnå et yderligere gain ved de laveste frekvenser ud fra placering af røret i kabinettet, samt positivt refleksionbidrag fra nærtliggende overflader. 


