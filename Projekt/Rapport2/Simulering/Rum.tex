\section{Rum}

Rummet der blev benyttet i projektet, er et mindre klasselokale på IHA, Katrinebjerg. 
Dette blev valgt grundet rummets neutralitet, med faste vægge, forholdsvist hårdt gulv, borde, stole og vinduer. Det tænkes derfor at være noget så nært neutrale forhold, så rummets efterklangstid kan estimeres med en simpel model. \\

Rummet er rektangulært, med størrelsen:

B=460cm \hspace{3cm}
L=580cm \hspace{3cm}
H=250cm\\

Ud fra Sabines formel, kan der estimeres en efterklangstid, hvor en rå absorbtionskoefficient tages i betragtning. 
Ved malede mursten bør absorptionskoefficienten $\alpha=0.015$ \cite{Attenuation} og efterklangstiden T60 være lig:

\(T_{60}=\frac{V}{S}*0.16=\frac{69.6}{0.3930}*0.16=28.33 ms\)

\(S=\alpha*(2*H+2*L+2*B)=0.3930\)

Denne efterklangstid er målt for valgte lokale i afsnit \ref{chapt:Maal}.
