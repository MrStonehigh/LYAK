\clearpage
\section{Rum}

Rummet der blev benyttet i projektet, er et mindre klasselokale på IHA, Katrinebjerg. 
Dette blev valgt grundet rummets neutralitet, med faste vægge, forholdsvist hårdt gulv, borde, stole og vinduer. Det tænkes derfor at være noget så nært neutrale forhold, så rummets efterklangstid kan estimeres med en simpel model. \\

Rummet er rektangulært, med størrelsen:

B=460cm \hspace{3cm}
L=580cm \hspace{3cm}
H=250cm\\

Da rummet er af mindre størrelse, kan efterklangstiden estimeres efter Sabines formel, hvor det blot er en rå antagelse af absorbtionskoefficienten der tages i betragtning. 

Ved malede mursten er absorptionskoefficienten tilnærmet $\alpha=0.015$ \cite{Attenuation} og efterklangstiden T60 vil være lig:

\begin{equation}\label{lign:T60}
	T_{60}=\frac{V}{S}*0.16=\frac{69.6}{0.3930}*0.16=28.33 ms
\end{equation}

Hvor
\begin{equation}
	S=\alpha*(2*H+2*L+2*B)=0.3930
\end{equation}


Denne efterklangstid er målt for valgte lokale i afsnit \ref{chapt:Maal}.
