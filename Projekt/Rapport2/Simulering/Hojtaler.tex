\section{Højtaler}


For at kunne sammenligne teori og praksis, kan højtaler-dataen fra tabel \ref{tab:TS} benyttes til at udformere et simuleret frekvensrespons, ud fra de forskellige portlængder på basrefleksen. 

Ved at beregne volumenhastigheden for den elektriske impedans, kan man opdele bidraget fra fronten af højtaleren, samt for porten, for til sidst at lave det samlede tryk til en ønsket afstand. 


\subsection{Volumenhastighed Front:}
\label{sec:sim_calc}

Ud fra impedansen af det samlede system, kan volumenhastigheden for højtalerenheden i kabinettet beregnes. Dette er hastigheden af det volumen af luft som skubbes foran højtaleren. 

{\Large\(qF=\)}{\huge \(\frac{F_A}{R_{AE}+s*M_{AS}+\frac{1}{s*C_{AS}}+Ras+\frac{1}{(s*C_{AB}+\frac{1}{s*M_{AP}}}}\) }

I denne beregning, indgår også \textit{massen af luften i porten} ($M_{AP})$, som beror sig på radius af porten og længden af porten:\\

\(M_{AP}=\frac{(\rho)}{S_P}*(L_P+1.46*\sqrt{\frac{S_P}{\pi}}\))

Da kabinettet i projektet allerede var konstrueret med en enkelt port, er radius af porten fastsat fra start.

\(r_P=0.025 m\)		\hspace{6.2cm} Portens Radius\\
\(S_P=\pi*r_P^2=pi*0.025^2=0.0020 m^2\)		\hspace{2cm} Portens overfladeareal\\

\(L_P=\frac{\gamma*P_0}{\rho*(2*\pi*f_p)^2*V_{B}}-1.46\sqrt{\frac{S_P}{\pi}}\)			\hspace{3cm} Hvor $\gamma \approx 1.4$ \& $P_0=100*10^3)$\\

Dermed får vi den optimale portlængde:\\

\(L_P=\frac{1.4*100*10^3}{1.18*(2*\pi*44)^2*16.5*10-3}-1.46\sqrt{\frac{0.0020}{\pi}}=15cm\)


\subsection{Volumenhastighed Port:}

Ovenpå at beregne volumenhastigheden for højtalerenheden, beregnes bidraget fra porten ud fra følgende formel:

{\Large\(qP=-qF*\)}{\huge \(\frac{\frac{1}{s*C_{AB}}}{\frac{1}{s*C_{AB}}+s*M_{AP}}\) }
\fxnote{s 61 Tores bog}


\subsection{Beregning af samlet lydtryk:}

For at omregne de to volumenhastigheder til et lydtryk i afstanden \textit{r}, benyttes følgende formel:

\(L=20*log_{10}(\frac{\frac{\rho*f}{r}*(qF+qP)}{pRef}) dB SPL\)
