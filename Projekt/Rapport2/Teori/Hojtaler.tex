\section{Højtaler}

%teori

En elektrodynamisk højtaler omsætter elektrisk energi til akustisk energi vha en svingspole som sætter højtalerens membran i svingninger som dermed frembringer lyd. Som model taler man oftest om en elektrisk model for selve svingspolens kobbermodstand og selvinduktion, en mekanisk model for membranens masse og udsvingets begrænsning, samt en akustisk model for lydens udstråling.
De tre modeller har en stor kobling idét der udveksles energi på tværs af modellerne. Dermed kan den impedansen for den elektriske model afspejle de andre modellers indvirkning på højtalerens Thiele-Small-parametre beskrevet i tabel \ref{tab:TS}.

Højtalerens elektrisk-mekaniske impedans er interessant, da man ud fra den kan sige noget om højtalerens frekvenskarakteristik. Impedansen vil have sit maksimum ved højtalerens resonansfrekvens, givet ved svingspolens selvinduktion samt højtalermembranens styr. For en basshøjtaler er resonansfrekvensen interessant, da man ved denne frekvens kan øge lydstyrken ved lave frekvenser, og dermed give lyden mere dybde og fylde.

Højtaleren der benyttes til projektet er en 6.5" mellemtone elektrodynamisk højtaler af mærket FW168\cite{FW168} fra firmaet Fountek \cite{Fountek}. 

På figur \ref{fig:kompletmodel} ses den komplette model for højtalerens elektriske, mekaniske og akustiske model.\citep{Elektroakustik} Komponentværdierne og forklaringen af disse, kan ses i tabel \ref{tab:TS}.

\begin{figure}[H]
	\centering
	\includegraphics[width=\textwidth]{Pics/kompletmodel.PNG}
	\label{fig:kompletmodel}
	\caption{Komplet model fro højtalerens elektrisk, mekaniske- og akustiske system. } 
\end{figure}

\begin{table}
	\centering
	\begin{tabular}[C]{|c|c|c|}
		
		
		\hline	
		\textbf{Thiele-Small parameter} & \textbf{Symbol} & \textbf{Værdi for FW168} \\\hline
		Svingspolens DC modstand & $R_E$ & $7.2\Omega$ \\\hline
		Svingspolens selvinduktion & $L_E$ & $1mH$  \\\hline
		Elektrisk godhed & $Q_{ES}$ & 0.452 \\\hline
		Masse af bevægeligt system & $M_{MS}$ & $14.7g$  \\\hline
		Eftergivelighed af styr & $C_{MS}$ & $0.821mm/N$  \\\hline
		Mekanisk godhed & $Q_{MS}$ & 3.246  \\\hline
		Mekanisk tabsmodstnd & $R_{MS}$ & \( \frac{1}{Q_{MS}}\sqrt{\frac{M_{MS}}{C_{MS}}}=1.304Ns/m \)  \\\hline
		Resonansfrekvens & $f_s$ & \( \frac{1}{2\pi\sqrt{M_{MS} C_{MS}}}=45.813Hz \) \\\hline
		
		
		Ækvivalent volumen & $ V_{AS} $ & $16.5L=0.017m^3$ \\\hline
		Kraftfaktor & $Bl$ & $8.2Tm$ \\\hline
		Membranens effektive areal & $S_D$ & $119cm^2$ \\\hline
		Maksimal lineær bevægelse & $X_{MAX}$ & $4.6mm\pm$ \\\hline
		
	\end{tabular}
	\label{tab:TS}
\end{table}

Omregner man modellen til en komplet elektrisk model, kan man udregne den elektriske impedans $Z_E$ for modellen. Denne impedans har et toppunkt ved højtalerens resonansfrekvens, og en minimumsværdi ved svingspolens $R_E$-værdi. 

Med værdierne fra tabel \ref{tab:TS}, som er opgivet i højtalerens datablad\cite{FW168}, opsættes en ligning for den elektriske impedans som funktion af frekvensen i ligning \ref{eq:eq1}

\begin{equation}\label{eq:eq1}
	Z_E(s)=R_E+sL_E+\frac{Bl^2}{\omega_s M_{MS}} \frac{ \omega_s s}{ s^2 + \frac{1}{Q_{MS}} \omega_s s + \omega_s^2} \end{equation} \begin{equation} \ \qquad  = 
	7.2\Omega + s \cdot 1mH + \frac{(8.2 Tm)^2}{287.8Hz \cdot 14.7gm} \frac{ 287.8Hz \cdot s}{ s^2 + \frac{1}{3.246} 287.8Hz \cdot s + (287.8Hz)^2}  \end{equation}

Impedansen vil være størst ved højtalerensresonansfrekvens $f_s$, som beregnes i ligning \ref{eq:fs}. Dette toppunkts maksimumværdi er givet ved ligning \ref{eq:Zmax} 

\begin{equation}\label{eq:fs}
	f_s=\frac{1}{2 \pi \sqrt{M_{MS} C_{MS}}}=45.813Hz
\end{equation}

\begin{equation}\label{eq:Zmax}
	Z_{max}=R_E+\frac{Bl^2}{R{MS}}=58.781\Omega
\end{equation}

På figur \ref{fig:ZE_graf} ses plottet af ligning \ref{eq:eq1} med værdierne for højtaleren. Kurveforløbet stemmer overens med det beregnede toppunkt $f_s$ og minimumsværdien $R_E$. Kurveforløbet stemmer ligeledes overens med det opgivne i databladet \citep{FW168}.

Ved frekvenser lavere end $f_s$ - det såkaldte fjederstyrede område - er impedansen domineret af svingspolens elektriske modstand, da membranens masse her er høj og styrernes eftergivelighed lav. Ved frekvenser højere end $f_s$ - det massestyrede område - er impedansen domineret af membranens masse som her er lav og svingspolens selvinduktion. Ved $f_s$ går den mekaniske masse og styr i resonans.


\begin{figure}[H]
	\centering
	\includegraphics[width=\textwidth]{Pics/ZE_graf.PNG}
	\label{fig:ZE_graf}
	\caption{Den elektriske impedans $Z_E$ som funktion af frekvensen} 
\end{figure}

Fra frekvensen $f_1$ bestemmes impedansen primært af svingspolens selvinduktion. $f_1$ er beregnet i ligning \ref{lig:f1}

\begin{equation}\label{lig:f1}
f_1=\frac{L_E}{2 \pi R_E}=1142Hz
\end{equation}

Det er altså ved de 45Hz som er højtalerens resonsnfrekvens, at vi vil forsøge at give et boost af lydniveauet.
