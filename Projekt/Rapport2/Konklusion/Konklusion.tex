\chapter{Konklusion}

På baggrund af samhørigheden mellem simulering og realiseringen for projektet, må forstærkningen af basgengivelsen for højtalerenheden siges at være en success. 
Sammenlignet med højtaleren i et lukket kabinet, var der et tydeligt højere gain i de laveste frekvenser, mens simuleringens inddrag af refleksionsbidraget viste sig at passe overordenligt godt sammen med de målte data. 

Til gengæld, viste det sig at der ikke var nogen åbenlys fordel i basrefleks-portens placering. Om noget, vil man kunne observere at der er en smule relativt højere gain, når porten er orienteret i samme retning som lytteren og højtalerenheden, men denne fordel er ikke så tydeligt udstående, at det kan konkluderes som en endelig forbedrende faktor.
Det giver dog god mening, da der selvfølgelig vil være mindre energi-tab i absorptionen af reflekterede overflader i den direkte vej, end hvis porten rammer en vinkel på næsten 90 grader. 

Afslutningsvist kan det konkluderes, at en optimal længde på basrefleks-porten kunne findes ud fra teorien og danne et velafstemt basrefleks-kabinet. Ydermere blev der skabt et gain, til trods for at den psyko-akustiske fordel vil kræve flere målrettede forsøg, for at kunne give en endelig konklusion på om kabinettets placering i rummet virkelig kan gøre en fordel i lytterens subjektive oplevelse. 

