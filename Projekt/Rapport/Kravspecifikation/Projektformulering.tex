\chapter{Projektformulering}

For at undgå akustisk kortslutning af en højtalerenhed, har man gennem historien fundet forskellige metoder til at modvirke dette fysiske fænomen. 
Den akustiske kortslutning fungerer ved de laveste frekvenser, hvor bølgelængden er meget stor ift. højtalerenhedens overfladeareal, og dermed er enheden ikke i stand til at gengive de laveste frekvenser.

Ved at sætte enheden i et kabinet, kan man undgå denne akustiske kortslutning i høj grad, og udnytte kabinettets egen-resonans til at behjælpe højtalerenhedens gengivelse af de lavere frekvenser. 

Denne resonans er yderigere behjulpet, ved brug af det princip som man kalder for "basreflex"; Et princip der benytter et rør i kabinettet, der kan danne yderligere resonanser og dermed opnå et højere gain ved de laveste frekvenser. En både billig og effektiv løsning, som opgradering til det lukkede kabinet. 

Vores projekt vil foruden at designe et sådan kabinet, gå på at eksperimentere med at opnå et yderligere gain i de laveste frekvenser vha. placering af røret i kabinettet, og placering af højtalerkabinetet i rum. 

Ideen er, at ramme et afstandsmæssigt "sweet-spot", hvor røret går yderligere i resonans med overfladen den er nær, og dermed giver mulighed for at forstærke basgengivelsen yderligere.  

