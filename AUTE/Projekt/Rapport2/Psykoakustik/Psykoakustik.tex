\chapter{Psykoakustik}

Menneskets øres udformning i både det ydre- mellem- og indre-øre, er anatomisk opbygget for at forstærke lyd, så man både kan høre \underline{hvad} det er for en lyd, men også \underline{hvor} lydkilden befinder sig. Ørerne er designet til at lokalisere lyd i 3 dimensioner, baseret på flere parametre;

\begin{itemize}
	\item  Den væsentligste parameter for at kunne identificere et objekt i 3D, er at man har to observationspunkter. I menneskets tilfælde med lyd, har vi et øre placeret på hver side af hovedet, som hver især kan dække en radius af ?? \fxnote{find ud af det - evt i Tores bog LS...Er det en ting? Radius er vel begrænset er kildens lydtryk? AB }. Hvis en lydkilde er placeret væk fra menneskets sagitale\fxnote{Sagitale? AB} plan, vil der være en tidsforskel på, hvornår lyden rammer hvert øre\fxnote{Ved lave frekvenser, AB}. Mere om det i afsnit \ref{sec:ITD}

\item  Når man hører en lyd, hører man ikke kun den direkte lyd, men også alle de reflektioner der er af lyden, fra det rum, eller de genstande, som lyden passerer undervejs fra sit udspring og til ørerne. Ud fra alle disse reflektioner danner hjernen sig et billede af hvilken type rum man står i, og hvor lyden kommer fra. \fxnote{Noget om T60 ift rumstørrelser og materialer - til efterklangstider LS}

\item  Også det ydre øres udformning, pinneaen, giver forskellige reflektioner\fxnote{overføringskarakteristikker? AB}, alt efter hvorfra lydkilden befinder sig, og ved hjælp af disse individuelle reflektioner, kan hjernen afgøre, hvor lydkilden befinder sig. Mere om dette i afsnit \ref{sec:HRTF}.


\end{itemize}

På visse punkter er menneskets hørelse dog også begrænset:

\begin{itemize}
	\item Ørets frekvensområde ændrer sig over tid. Spædbørn kan høre fra omkring 20Hz - 20kHz, mens en 80-årig måske kun kan høre frekvenser fra 50Hz - 15kHz \fxnote{Find ref LS}. Alle frekvenser bliver heller ikke forstærket lige godt inde i øret. Det vil sige at der skal et relativt større lydtryk til at høre helt dybe eller helt lyse frekvenser, end frekvenser i mellemtoneområdet. Se figure \ref{fig:pressure}
	
	\figur{.8}{pressure}{Forholdet mellem SPL og frekvens ved forskellige phon-styrker}{pressure} 
	
	\item Det kan være svært at angive præcist hvor en lydkilde er\fxnote{(, på den præcise decimal,) Udelad? AB} hvis man ikke når at bevæge hovedet inden lyden er væk igen. Dette kaldes "Cone of Confusion", og er et stort problem i gengivelsen af 3D lyd i høretelefoner. Mere om det i afsnit \ref{sec:3DBasics}. I den virkelige verden lægger man ofte ikke mærke til det, da man vil nå at dreje hovedet, så hjerne kan registrere lyden i to forskellige positioner, og dermed nå at registrere præcist hvor lydkilden befinder sig.
\end{itemize}