\chapter{Psykoakustik}

Menneskets ører er designet til at høre lyd i 3 dimensioner på flere parametre;

{\Large  1)}\\
Den væsentligste parameter for at kunne identificere et objekt i 3D, er at man har to observationspunkter. I menneskets tilfælde med lyd, har vi et øre placeret på hver side af hovedet, som hver især kan dække en radius af ?? \fxnote{find ud af det - evt i Tores bog LS}. Hvis en lydkilde er placeret væk fra menneskets sagitale plan, vil der være en tidsforskel på, hvornår lyden rammer hvert øre. Mere om det i sektion \ref{sec:ITD}

{\Large  2)}\\
Ørets udformning i både det ydre- mellem- og indre-øre, er anatomisk opbygget for at forstærke lyden, både så man kan høre \underline{hvad} det er for en lyd, men også \underline{hvor} lydkilden befinder sig. Når man hører en lyd, hører man ikke kun den direkte lyd, men også alle de reflektioner der er af lyden, fra det rum, eller de genstande, som lyden passere undervejs fra sit udspring og til ørerne. Også det ydre øres udformning giver forskellige reflektioner, alt efter hvorfra lydkilden befinder sig, og ved hjælp af disse reflektioner, kan hjernen afgøre, hvor lydkilden befinder sig. Mere om dette i sektion \ref{sec:HRTF}.