\section{Head Related Transfer Function}
\label{sec:HRTF}

Head Related Transfer Function (HRTF) er betegnelsen for den overføringsfunktion, som IID og det ydre øres udformning udgør. Mange forskerhold har gennem årene arbejdet med at optage og genskabe HRTF'er for mange forskellige azimuth- og elevation-punkter, ved at måle og optage den impulsrespons som øret oplever, når impulsen opstår i en given azimuth og elevation. Nogle af disse databaser af HRTF'er er blevet standardiseret og samlet under ét, kaldet SOFA (Spatially Oriented Format for Acoustics) \cite{SOFA}. Til dette projekt er anvendt databasen fra MIT \cite{MIT} som med 710 målepunkter, dækker elevation fra -40$^\circ$ til 90$^\circ$ i en komplet 360$^\circ$ azimuth i en distance på 1 meter fra en KEMAR som det er vist på figur \ref{fig:mit}.

\figur{.8}{mit}{HRTF målepunkter for MIT KEMAR}{mit}

 På figur \ref{fig:hrtfgraf} ses forskellen på en HRTF for hhv højre og venstre øre for en impuls i positionen azimuth = 270$^\circ$ og elevation = 10$^\circ$. Da graferne er skaleret ens er det tydeligt at se, at impulsresponsen for det venstre øre, er væsentligt kraftigere end for det højre øre, og at impulsresponset for det højre øre er mere forsinket end det for det venstre. \footnote{Forskellen i delayet mellem de to impulsresponser er 0.6ms, hvilket svarer til KEMAR'ens hovedstørrelse på 23.4cm}, hvilket altsammen giver mening, da azimuth = 270$^\circ$ er direkte ud for det venstre øre. Kigger man på DFT-transformationen af impulsresponserne ses det også tydeligt, at IID spiller væsentligt ind for det højre øre, hvor de høje frekvenser bliver dæmpet betydeligt pga. hovedets akustiske skygge.

\figur{1}{HRTF_grafer.png}{Fire grafer fra en impuls i positionen azimuth = 270$^\circ$ og elevation = 10$^\circ$. a) Impulsresponsen for højre øre. b) impulsresponsen for venstre øre. c) DFT transformerede impulsrespons for højre øre. d) DFT transformerede impulsrespons for venstre øre. }{hrtfgraf}