\section{Afstandsgengivelse}

For at give lytteren en fornemmelse af afstanden til lydkilden, er det nødvendigt at ændre i lydniveauet efter afstanden der ønskes oplevet. \\ 
Dette element kan dog være besværligt at gengive overbevisende, da mennesket automatisk vægter sin forforståelse af en lydkilde ved bedømmelse af afstand. Ved tale, vil man altså være væsentligt skarpere til at bedømme om afstanden opleves som ønsket, imod et mere atypisk, ukendt element af lyd\cite{SpatialBook}.

For at give fornemmelsen af afstand mellem lyttepositionen og lydkilden benyttes afstandsreglen som siger, at lydtrykket aftager med $\frac{1}{r}$, hvor r = distancen mellem lyttepositionen og lydkilden. Det betyder at øger man distancen mellem de to positioner til den dobbelte distance, vil lydtrykket falde til det halve. Omregnet i decibel falder lydtrykket 6dB for hver fordobling af distancen. Ligeledes øges lydtrykket når distancen falder. \fxnote{Reference? Er det også sådan i 3dlyd? AB}