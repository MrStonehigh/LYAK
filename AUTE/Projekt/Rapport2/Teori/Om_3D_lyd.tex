%\section{Digital lyd i 3D}
%\label{sec:3DBasics}
%
%For at gengive lyd i 3 dimensioner kræves der som nævnt to lydkilder, hvor det er amplitude- og faseforskellen mellem de to lydkilder som i overvejende grad giver fornemmelsen af 3 dimensioner\fxnote{De parametre giver kun umiddelbart fornemmelse for position horisontalt? AB}. Man kan overveje om det vil lykkes at afspille det på et normalt stereoanlæg, men med den teknologi som der kendes til i dag, giver dette ikke et godt resultat. Grunden til det, er at personen som skal lytte, skal stå meget præcist og meget stille mellem de to højtalere for at det ville lykkedes, og det endda uden at medregne alle de forstyrrende reflektioner der ville være i selve lytterummet.\fxnote{Mere nøjagtigt, flere fagord: Cross-talk cancellation osv.}
%
%Digital 3D-lyd virker bedst ved at lytte til det i et sæt høretelefoner, hvor afstanden mellem de to lydkilder (hhv højre og venstre side af høretelefonen) er fastlåst ift. afstanden til ørerne. \fxnote{Samt der ikke foregår nogen forstyrrelser fra venstre til højre og omvendt, AB}
%
%Der er tre parametre\fxnote{Vi bruger parametre både om spatiale cues og om lokaliseringsparametre... AB} som skal fastsættes for at kunne placere et lydobjekt i et 3-dimensionelt rum; Den horisontale vinkel (kaldet azimuth), den vertikale vinkel (kaldet elevation) og distancen fra lyttepositionen til lydkilden. 
%
%
%
