%\section{Interaural Time Difference\\ \& Interaural Intensity Difference}
%\label{sec:ITD}
%\fxnote{Section titel ' Spatiale Cues ' istedet? AB}
%Følgende beskrivelse er baseret på en lydkilde som ligger væk fra kroppens sagitale plan, dvs. i en azimuth $ \ne 0^\circ$. 
%
%
%
%\subsection{IID}
%
%Interaural Intensity Difference (IID), eller Interaural Level Difference (ILD) er forskellen i lydsignalets amplitude ved hhv højre og venstre øre hhv før og efter påvirkning fra den akustiske skygge. Pga den akustiske skygge, vil lydsignalets amplitude blive dæmpet, og ud fra denne amplitude forskel mellem højre og venstre øre, kan hjernen afgøre lydkildens azimuth og elevation.
%
%Overlappet mellem om hjernen reagerer mest på ITD og IID ligger - som vist på figur \ref{fig:itdiid} - omkring 1kHz-2kHz  afhængig af den eksakte azimuth og elevation.
%
%I dette projekt er IID integreret i hrtf'erne som beskrives i afsnit \ref{sec:HRTF}.
%

