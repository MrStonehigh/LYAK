\chapter{Konklusion}

Ud fra både lyttetesten samt projektdeltagernes egne bedømmelser, er der lang vej igen for at lave 3-dimensionel lyd, som virker optimalt for alle personer. De anvendte spatial cues som er anvendt i dette projekt, giver en fin fornemmelse af lyd som flyttes rundt i rummet, men for at opnå enighed om den nøjagtige position kræves flere specifikke beregninger. 
Specielt ITD og ILD er meget frekvensafhængigt, og kræver derfor yderligere, eller andre, former for algoritmer, end hvad der er anvendt i dette projekt. HRTF databasen fra MIT, giver også primært en fordel for de personer, hvis ydre øres udformning er inkluderet i materialet til databasen.
Opfattelsen af distance, ville kunne gøres mere præcis ved at tilføje rumklang for at angive distancen bedre. Ligeledes ville dette kunne hjælpe på opfattelsen af elevation, som i et sæt lukkede høretelefoner, kan være svær at placere korrekt.

Næste skridt for dette projekt ville være, at implementere systemet enten i forbindelse med virtual eller augmented reality, hvorved en lydkilde ville kunne fastlåses i en position selvom testpersonen bevægede sit hoved fra side til side eller op og ned. Også påvirkningen af synssansen ville give et ekstra boost til opfattelsen af lydens placering.

Projeket har dog vist sig at fungere godt med den opbyggede GUI, hvorved der hurtigt og nemt kunne flyttes rundt på lydkilden i realtid med både en visuel og auditiv feedback. 

Projektdeltagerne er ligeledes blevet gjort klogere på både akustik og digital gengivelse af lyd, samt de mange udfordringer der ligger i dette område.