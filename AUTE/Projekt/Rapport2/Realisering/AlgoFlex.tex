\section{AlgoFlex}
\label{sec:AlgoFlex}

AlgoFlex er et realtidsscript udviklet hos MUS!C GROUP ved TC Electronic til intern test af realtidsopstillinger. Det er et server-baseret script, hvor man kan indsætte forskellige blokke og forbinde dem som man ønsker til eksekvering i bl.a. Matlab. Det er også muligt at oprette sine egne blokke, hvilket bl.a. er gjort med Fx, blokken\fxnote{Navngiv som Convolution blok, AB} som er en simpel foldningsblok. Resten af blokkene anvendt i dette projekt er standardblokke fra AlgoFlex-biblioteket.

\figur{.9}{3DAUTE}{Blokdiagram over projektets AlgoFlex-blokke}{3DAUTE}

På figur \ref{fig:3DAUTE}\fxnote{Lav simpelt blokdiagram over algoflexblokkene LS}\fxfatal{eller brug tid på at lave pænere AlgoFlexDiagram! LS} ses opbygningen af AlgoFlex-blokkene. 

\subsection{Blokbeskrivelse}

\begin{labeling}{alligatoralligatorallig}
	\item[\textbf{MatrixPlayer}] Er den blok som indeholder det ønskede musiksignal, og som afspiller det i repeat. Herfra deles signalet i i to, hvor udgang 1 er venstre kanal, og udgang 2 er højre kanal.
	\item[\textbf{Delay}] Blokken som modtager det beregnede tidsdelay for hvert øre som parameter, og derfra forsinker signalet som ønsket.
	\item [\textbf{Gain}] Når distancen er udregnet, udregnes det ønskede gain ud fra afstandsreglen, som sættes som Gain-blokkens parameter. 
	%	\item[\textbf{AEC}] Acoustic Echo Cancelation. Teknik til at bortfiltrere et akustisk ekko. \fxnote{lidt bedre forklaring - Bør vi ikke slette den? Vi bruger det ikke som sådan længere, AB}
	\item[\textbf{Fx}] \fxnote{Navngiv Convolution, AB}En selvkodet blok som foretager en foldning af inputsignalet og de filterkoefficienter fra HRTF'erne der gives som parametre. 
	
	\item[\textbf{AudioStream}] Output blokken som sender signalet ud i hhv venstre og højre kanal vha PC'ens normale lydkort. 
	\item[\textbf{MatrixRecorder}] To MatrixRecordere benyttes til at opsamle data løbende fra blokkene, og er blandt andet benyttet både til debugging, og til plottene på figur \ref{fig:blok_output}.
	
\end{labeling}



