\chapter{Projektformulering}

I en verden hvor Virtual Reality og Augmented Reality er i hastig fremdrift, er det vigtigt at kunne stimulere alle sanser for at gøre oplevelsen så realistisk som muligt. Synsoplevelsen har haft det primære fokus for udviklere verden over, men også lytteoplevelsen er i hastig fremdrift. Her er det ikke godt nok med den den velkendte stereolyd i høretelefonerne hvis man skal tro på at man står i et "ægte" virtuelt rum. Her er der behov for også at kunne placere lydobjekter så man udelukkende ved hjælp af hørelsen kan regne ud, hvor objektet befinder sig i rummet.\\ Dette projekt vil opbygge en realtidsalgoritme til at teste hvor godt man digitalt kan gengive oplevelsen af et lydobjekts placering i 3 dimensioner. Ved hjælp af en lyttetest, vil der drages en konklusion på, hvor realistisk lydbilledet tager sig ud.

%\textbf{Hvor godt kan man realisere et 3-Dimensionelt lydbillede digitalt i et sæt almindelige høretelefoner?}

\subsection{TAK}

En tak skal lyde til TC Electronic for lån af AlgoFlex systemet til realtidsopbygning af dette projekt.

